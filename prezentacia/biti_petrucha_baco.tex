\documentclass{beamer}
\usepackage[utf8]{inputenc}
\usepackage{slovak}
\usepackage{graphicx}
\usepackage{ragged2e}
\usepackage{verbatim}
\usetheme{Warsaw}

\title{Experimentálne porovnanie výkonu OpenVPN pri~použití rôznych kryptografických algoritmov}
\author[Petrucha, Bačo]{Michal Petrucha, Ladislav Bačo\\{\tiny RNDr. Jaroslav Janáček, PhD.}}
\institute{
Fakulta matematiky, fyziky a informatiky\\
Univerzita Komenského, Bratislava
}
\subject{Informatika}
\date{26. mája 2014}

\begin{document}

\frame{\titlepage}

\begin{frame}
	\frametitle{Zadanie projektu}
	\begin{block}{Úloha}
	\justifying
	Cieľom projektu je experimentálne zistiť vplyv použitého kryptografického algoritmu na maximálnu prenosovú rýchlosť dosiahnuteľnú medzi dvomi počítačmi pri nasadení OpenVPN, resp. vplyv na vyťaženia procesora v prípade, že dosiahnutá maximálna prenosová rýchlosť bude limitovaná inými faktormi. Na riešenie projektu je možné využiť sieťové laboratórium KI.
	\end{block}
\end{frame}

\begin{frame}[fragile]
	\frametitle{OpenVPN}
	\framesubtitle{Konfigurácia a sprevádzkovanie}
	\begin{itemize}
		\item statický kľúč vs. certifikáty, privátne a verejné kľuče
			\begin{itemize}
				\item výhody: jednoduchšie nastavenie, žiadne PKI
			\end{itemize}
		\item minimálna konfigurácia:
			\begin{columns}
				\tiny
				\column[t]{0.4\textwidth}
					\begin{block}{server}
\begin{verbatim}
dev tun
ifconfig $SERVER_VPN_IP $CLIENT_VPN_IP
secret $STATIC_KEY
\end{verbatim}
					\end{block}
				\column[t]{0.4\textwidth}
					\begin{block}{client}
\begin{verbatim}
dev tun
remote $SERVER_IP
ifconfig $CLIENT_VPN_IP $SERVER_VPN_IP
secret $STATIC_KEY
\end{verbatim}
					\end{block}
			\end{columns}

		\item ďalšie parametre:
			\begin{itemize}
				\item HMAC autentifikácia paketov: \texttt{auth SHA512}
				\item šifrovací algoritmus: \texttt{cipher AES-256-CBC}
				\item veľkosť kľúča: \texttt{keysize n}
				\item kompresia: \texttt{comp-lzo yes|no}
			\end{itemize}
	\end{itemize}
\end{frame}

\begin{frame}
	\frametitle{OpenVPN}
	\framesubtitle{Testujeme\dots}
	\begin{itemize}
		\item \texttt{nc, pv}
		\item gigabit ethernet, bez OpenVPN
			\begin{itemize}
				\item \texttt{/dev/zero} $\approx$ 80 MBps
				\item \texttt{/dev/urandom} $\approx$ 5 MBps ??
				\item \texttt{video.mp4} $\approx$ 50 MBps
				\item bottleneck: HDD $\rightarrow$ ramdisk
			\end{itemize}
		\bigskip
		\item s OpenVPN:
			\begin{itemize}
				\item 40 šifrovacích algoritmov
				\item 24 hašovacích funkcií
				\item kompresia
				\item aspoň dva druhy súborov (nulový, náhodný)
				\item obojsmerné posielanie
				\item $40 \cdot 24 \cdot 2 \cdot 2 \cdot 2 = 7680$ testov!!
			\end{itemize}
		\item to určite nechceme robiť ručne\dots
	\end{itemize}
\end{frame}

\begin{frame}
	\frametitle{Testovací skript}
	\framesubtitle{Inicializácia}
	\begin{itemize}
		\item ramdisk, adresáre pre test
		\item náhodný súbor z \texttt{dev/urandom}, odoslanie klientovi\\
			 nulový súbor z \texttt{/dev/zero}
		\item generovanie kľúča, odoslanie klientovi
		\item vytvorenie, sychnronizácia a spoločná podmnožina použiteľných šifier a hašov\\
			\bigskip
			{\small
			\texttt{openvpn --show-ciphers}, \texttt{openvpn --show-digests}\\
			\texttt{sort \$TEST\_DIR/ciphers.* | uniq -d | awk '\{print \$1\}' > \$CIPHERS}}
	\end{itemize}
\end{frame}

\begin{frame}[fragile]
	\frametitle{Testovací skript}
	\framesubtitle{Testovanie OpenVPN}
	\begin{itemize}
		\item TODO Testovacie prostredie:
			\begin{itemize}
				\item Debian GNU/Linux 7.5 (wheezy), HW?
				\item openvpn --version
				\item echo "version" | openssl
			\end{itemize}
		\item
{\small
\begin{verbatim}
for COMP in no yes; do
    for DIGEST in $(cat $DIGESTS); do
        for CIPHER in $(cat $CIPHERS); do
\end{verbatim}
}
		\item vygenerovanie konfiguračných súborov, štart OpenVPN
		\item posielanie súborov z klienta na server a zo servera na klienta
{\tiny
\begin{verbatim}
BEFORE=$(get_time)
nc $(get_peer_ip) $NC_PORT < $1
AFTER=$(get_time)
echo "$AFTER - $BEFORE" | bc -l
\end{verbatim}
}
			\begin{itemize} 
				\item fungujú iba CBC šifry!
				\item neukončovanie \texttt{nc} listenera $\rightarrow$ sleep
			\end{itemize}	

	\end{itemize}
\end{frame}

\begin{frame}
	\frametitle{Výsledky}
	\begin{itemize}
		\item čas behu: ?? dní, prenesených 1000 GB v 3200 testoch
		\item 
	\end{itemize}
\end{frame}


\begin{frame}
	\frametitle{Záver}
	\begin{itemize}
		\item TODO
	\end{itemize}

	\bigskip
	Odkazy:
	\small
	\begin{itemize}
		\item \href{https://github.com/laciKE/openvpn\_test/}{https://github.com/laciKE/openvpn\_test/}
		\item \href{https://openvpn.net/index.php/open-source/documentation/miscellaneous/78-static-key-mini-howto.html}{https://openvpn.net/index.php/open-source/documentation/miscellaneous/78-static-key-mini-howto.html}
	\end{itemize}
\end{frame}



\end{document}
