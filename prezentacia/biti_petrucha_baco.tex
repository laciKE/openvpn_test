\documentclass{beamer}
\usepackage[utf8]{inputenc}
\usepackage{slovak}
\usepackage{graphicx}
\usepackage{ragged2e}
\usepackage{verbatim}
\usetheme{Warsaw}

\title{Experimentálne porovnanie výkonu OpenVPN pri~použití rôznych kryptografických algoritmov}
\author[Petrucha, Bačo]{Michal Petrucha, Ladislav Bačo\\{\tiny RNDr. Jaroslav Janáček, PhD.}}
\institute{
Fakulta matematiky, fyziky a informatiky\\
Univerzita Komenského, Bratislava
}
\subject{Informatika}
\date{26. mája 2014}

\begin{document}

\frame{\titlepage}

\begin{frame}
	\frametitle{Zadanie projektu}
	\begin{block}{Úloha}
	\justifying
	Cieľom projektu je experimentálne zistiť vplyv použitého kryptografického algoritmu na maximálnu prenosovú rýchlosť dosiahnuteľnú medzi dvomi počítačmi pri nasadení OpenVPN, resp. vplyv na vyťaženia procesora v prípade, že dosiahnutá maximálna prenosová rýchlosť bude limitovaná inými faktormi. Na riešenie projektu je možné využiť sieťové laboratórium KI.
	\end{block}
\end{frame}

\begin{frame}[fragile]
	\frametitle{OpenVPN}
	\framesubtitle{Konfigurácia a sprevádzkovanie}
	\begin{itemize}
		\item statický kľúč vs. certifikáty, privátne a verejné kľuče
			\begin{itemize}
				\item výhody: jednoduchšie nastavenie, žiadne PKI
			\end{itemize}
		\item minimálna konfigurácia:
			\begin{columns}
				\tiny
				\column[t]{0.4\textwidth}
					\begin{block}{server}
\begin{verbatim}
dev tun
ifconfig $SERVER_VPN_IP $CLIENT_VPN_IP
secret $STATIC_KEY
\end{verbatim}
					\end{block}
				\column[t]{0.4\textwidth}
					\begin{block}{client}
\begin{verbatim}
dev tun
remote $SERVER_IP
ifconfig $CLIENT_VPN_IP $SERVER_VPN_IP
secret $STATIC_KEY
\end{verbatim}
					\end{block}
			\end{columns}

		\item ďalšie parametre:
			\begin{itemize}
				\item HMAC autentifikácia paketov: \texttt{auth SHA512}
				\item šifrovací algoritmus: \texttt{cipher AES-256-CBC}
				\item kompresia: \texttt{comp-lzo yes|no}
			\end{itemize}
	\end{itemize}
\end{frame}

\begin{frame}
	\frametitle{OpenVPN}
	\framesubtitle{Testujeme\dots}
	\begin{itemize}
		\item \texttt{nc, pv}
		\item gigabit ethernet, bez OpenVPN
			\begin{itemize}
				\item \texttt{/dev/zero} $\approx$ 112 MBps
				\item \texttt{/dev/urandom} $\approx$ 5 MBps ??
				\item \texttt{video.mp4} $\approx$ 80 MBps
				\item bottleneck: HDD $\rightarrow$ ramdisk
			\end{itemize}
		\bigskip
		\item s OpenVPN:
			\begin{itemize}
				\item 58 šifrovacích algoritmov
				\item 25 hašovacích funkcií
				\item kompresia
				\item aspoň dva druhy súborov (nulový, náhodný)
				\item obojsmerné posielanie
				\item $58 \cdot 25 \cdot 2 \cdot 2 \cdot 2 = 11600$ testov!!
			\end{itemize}
		\item to určite nechceme robiť ručne\dots
	\end{itemize}
\end{frame}

\begin{frame}
	\frametitle{Testovací skript}
	\framesubtitle{Inicializácia}
	\begin{itemize}
		\item ramdisk, adresáre pre test
		\item náhodný súbor z \texttt{/dev/urandom}, odoslanie klientovi\\
			 nulový súbor z \texttt{/dev/zero}
		\item generovanie kľúča, odoslanie klientovi
		\item vytvorenie, sychnronizácia a spoločná podmnožina použiteľných šifier a hašov\\
			\bigskip
			{\small
			\texttt{openvpn --show-ciphers}, \texttt{openvpn --show-digests}\\
			\texttt{sort \$TEST\_DIR/ciphers.* | uniq -d | awk '\{print \$1\}' > \$CIPHERS}}
	\end{itemize}
\end{frame}

\begin{frame}[fragile]
	\frametitle{Testovací skript}
	\framesubtitle{Testovanie OpenVPN}
	\begin{itemize}
		\item Testovacie prostredie:
			\begin{itemize}
				\item Debian GNU/Linux 7.4, Intel Pentium 4@2.80GHz, 1GB RAM
				\item OpenVPN 2.2.1
				\item OpenSSL 1.0.1e-2+deb7u6
				\item 1Gbps sieť
			\end{itemize}
		\item
{\small
\begin{verbatim}
for COMP in no yes; do
    for DIGEST in $(cat $DIGESTS); do
        for CIPHER in $(cat $CIPHERS); do
\end{verbatim}
}
		\item vygenerovanie konfiguračných súborov, štart OpenVPN
		\item posielanie súborov z klienta na server a zo servera na klienta
{\tiny
\begin{verbatim}
BEFORE=$(get_time)
nc -q 0 $(get_peer_ip) $NC_PORT < $1
AFTER=$(get_time)
echo "$AFTER - $BEFORE" | bc -l
\end{verbatim}
}
			\begin{itemize} 
				\item fungujú iba CBC šifry!
				\item neukončovanie \texttt{nc} listenera $\rightarrow$ sleep
			\end{itemize}	

	\end{itemize}
\end{frame}

\begin{frame}
	\frametitle{Výsledky}
	\begin{itemize}
		\item 16 šifier
		\item 25 hašov
		\item 3200 testov
		\item 1000 GB
		\item 30 hodín
		\item spracovanie meraní pomocou Pythonu a SQLite
	\end{itemize}
	\begin{table}
		\begin{tabular}{|l||l|l|l|l|l|}
			\hline
			\bf súbor & \bf  avg & \bf stdev & \bf median & \bf min & \bf max \\ \hline 
			random & 1.0027 & 0.002 & 1.0027 & 0.9796 & 1.0075 \\ \hline
			zero & 0.6614 & 0.0943 & 0.6627 & 0.5164 & 0.84\\ \hline
		\end{tabular}
		\caption{Vplyv kompresie na čas prenosu; čas prenosu bez kompresie: 1}
	\end{table}
\end{frame}

\begin{frame}
	\frametitle{Výsledky}
	\begin{table}
		\footnotesize 
		\begin{tabular}{|l||l|l|l|l|l|}
			\hline
			\bf šifra & \bf  avg & \bf stdev & \bf median & \bf min & \bf max \\ \hline 
BF-CBC & 1.0 & 0.0 & 1.0 & 1.0 & 1.0\\ \hline
CAMELLIA-128-CBC & 1.0148 & 0.0036 & 1.014 & 1.0096 & 1.0236 \\ \hline
CAMELLIA-256-CBC & 1.0573 & 0.0044 & 1.0583 & 1.0456 & 1.0655 \\ \hline
CAMELLIA-192-CBC & 1.058 & 0.0046 & 1.0599 & 1.0467 & 1.0663 \\ \hline
CAST5-CBC & 1.0781 & 0.0078 & 1.0799 & 1.0603 & 1.0894 \\ \hline
DES-CBC & 1.1026 & 0.0119 & 1.1034 & 1.0776 & 1.1402 \\ \hline
SEED-CBC & 1.1045 & 0.0068 & 1.1058 & 1.0852 & 1.1147 \\ \hline
DESX-CBC & 1.1134 & 0.0109 & 1.1159 & 1.0887 & 1.1321 \\ \hline
AES-128-CBC & 1.2823 & 0.0221 & 1.2867 & 1.2235 & 1.3078 \\ \hline
AES-192-CBC & 1.3687 & 0.0293 & 1.376 & 1.2925 & 1.4016 \\ \hline
RC2-64-CBC & 1.4211 & 0.0364 & 1.4294 & 1.3309 & 1.4679 \\ \hline
RC2-40-CBC & 1.4213 & 0.0368 & 1.4306 & 1.3314 & 1.4694 \\ \hline
RC2-CBC & 1.4213 & 0.0365 & 1.4315 & 1.3318 & 1.4687 \\ \hline
AES-256-CBC & 1.4553 & 0.0376 & 1.4671 & 1.3593 & 1.498 \\ \hline
DES-EDE-CBC & 1.5821 & 0.0508 & 1.5934 & 1.4585 & 1.6539 \\ \hline
DES-EDE3-CBC & 1.5824 & 0.0504 & 1.592 & 1.4599 & 1.6551 \\ \hline
		\end{tabular}
		\caption{Vplyv šifrovacieho algoritmu na čas prenosu; čas prenosu pri použití BF-CBC: 1}
	\end{table}
\end{frame}

\begin{frame}
	\frametitle{Výsledky}
	\begin{table}
		\tiny 
		\begin{tabular}{|l||l|l|l|l|l|}
			\hline
			\bf haš & \bf  avg & \bf stdev & \bf median & \bf min & \bf max \\ \hline 
MD4 & 1.0 & 0.0 & 1.0 & 1.0 & 1.0 \\ \hline
RSA-MD4 & 1.0027 & 0.0066 & 1.0013 & 0.9985 & 1.0272 \\ \hline
MD5 & 1.0082 & 0.0019 & 1.0084 & 1.0039 & 1.0111 \\ \hline
RSA-MD5 & 1.0092 & 0.002 & 1.0092 & 1.005 & 1.0114 \\ \hline
ecdsa-with-SHA1 & 1.0494 & 0.0098 & 1.0509 & 1.0312 & 1.0643 \\ \hline
DSA-SHA1-old & 1.0498 & 0.0094 & 1.0512 & 1.0328 & 1.0631 \\ \hline
SHA1 & 1.0502 & 0.0094 & 1.0507 & 1.032 & 1.0627 \\ \hline
DSA-SHA & 1.0505 & 0.0099 & 1.0523 & 1.0328 & 1.0649 \\ \hline
RSA-SHA1-2 & 1.0505 & 0.0095 & 1.0525 & 1.033 & 1.0656 \\ \hline
RSA-SHA1 & 1.0506 & 0.0097 & 1.0524 & 1.0324 & 1.065 \\ \hline
DSA-SHA1 & 1.0507 & 0.0102 & 1.051 & 1.0327 & 1.0654 \\ \hline
DSA & 1.0509 & 0.0095 & 1.052 & 1.0336 & 1.0644 \\ \hline
RIPEMD160 & 1.0741 & 0.014 & 1.0746 & 1.0492 & 1.0936 \\ \hline
RSA-RIPEMD160 & 1.0746 & 0.0147 & 1.0761 & 1.0503 & 1.0949 \\ \hline
RSA-SHA & 1.0846 & 0.0161 & 1.0873 & 1.0574 & 1.1073 \\ \hline
SHA & 1.0857 & 0.0165 & 1.0863 & 1.0582 & 1.1101 \\ \hline
RSA-SHA224 & 1.133 & 0.0246 & 1.1346 & 1.0907 & 1.1695 \\ \hline
SHA224 & 1.1331 & 0.0241 & 1.1346 & 1.0911 & 1.1687 \\ \hline
SHA256 & 1.1367 & 0.0254 & 1.1369 & 1.0942 & 1.1735 \\ \hline
RSA-SHA256 & 1.1369 & 0.0253 & 1.1369 & 1.0944 & 1.174 \\ \hline
RSA-SHA512 & 1.2056 & 0.0383 & 1.2098 & 1.1504 & 1.2561 \\ \hline
SHA512 & 1.206 & 0.0379 & 1.2113 & 1.1506 & 1.2593 \\ \hline
SHA384 & 1.2106 & 0.0397 & 1.2164 & 1.1521 & 1.2651 \\ \hline
RSA-SHA384 & 1.2108 & 0.0396 & 1.2163 & 1.1525 & 1.2635 \\ \hline
whirlpool & 1.327 & 0.0589 & 1.3344 & 1.2401 & 1.4065 \\ \hline
		\end{tabular}
		\caption{Vplyv hašovacej funkcie na čas prenosu; čas prenosu pri použití MD4: 1}
	\end{table}
\end{frame}



%SELECT data_file, ROUND(AVG(delta), 4), ROUND(STDEV(delta), 4), ROUND(MEDIAN(delta), 4), ROUND(MIN(delta), 4), ROUND(MAX(delta), 4) FROM compress_relative GROUP BY data_file;
%random            1.0027    0.002     1.0027    0.9796    1.0075  
%zero              0.6614    0.0943    0.6627    0.5164    0.84 


%SELECT cipher, ROUND(AVG(relative_duration), 4), ROUND(STDEV(relative_duration),4), ROUND(MEDIAN(relative_duration), 4), ROUND(MIN(relative_duration), 4), ROUND(MAX(relative_duration), 4) FROM cipher_speedup GROUP BY cipher ORDER BY AVG(relative_duration);
%BF-CBC            1.0       0.0       1.0       1.0       1.0     
%CAMELLIA-128-CBC  1.0148    0.0036    1.014     1.0096    1.0236  
%CAMELLIA-256-CBC  1.0573    0.0044    1.0583    1.0456    1.0655  
%CAMELLIA-192-CBC  1.058     0.0046    1.0599    1.0467    1.0663  
%CAST5-CBC         1.0781    0.0078    1.0799    1.0603    1.0894  
%DES-CBC           1.1026    0.0119    1.1034    1.0776    1.1402  
%SEED-CBC          1.1045    0.0068    1.1058    1.0852    1.1147  
%DESX-CBC          1.1134    0.0109    1.1159    1.0887    1.1321  
%AES-128-CBC       1.2823    0.0221    1.2867    1.2235    1.3078  
%AES-192-CBC       1.3687    0.0293    1.376     1.2925    1.4016  
%RC2-64-CBC        1.4211    0.0364    1.4294    1.3309    1.4679  
%RC2-40-CBC        1.4213    0.0368    1.4306    1.3314    1.4694  
%RC2-CBC           1.4213    0.0365    1.4315    1.3318    1.4687  
%AES-256-CBC       1.4553    0.0376    1.4671    1.3593    1.498   
%DES-EDE-CBC       1.5821    0.0508    1.5934    1.4585    1.6539  
%DES-EDE3-CBC      1.5824    0.0504    1.592     1.4599    1.6551



%SELECT digest, ROUND(AVG(relative_duration), 4), ROUND(STDEV(relative_duration),4), ROUND(MEDIAN(relative_duration), 4), ROUND(MIN(relative_duration), 4), ROUND(MAX(relative_duration), 4) FROM digest_speedup GROUP BY digest ORDER BY AVG(relative_duration);
%MD4               1.0       0.0       1.0       1.0       1.0     
%RSA-MD4           1.0027    0.0066    1.0013    0.9985    1.0272  
%MD5               1.0082    0.0019    1.0084    1.0039    1.0111  
%RSA-MD5           1.0092    0.002     1.0092    1.005     1.0114  
%ecdsa-with-SHA1   1.0494    0.0098    1.0509    1.0312    1.0643  
%DSA-SHA1-old      1.0498    0.0094    1.0512    1.0328    1.0631  
%SHA1              1.0502    0.0094    1.0507    1.032     1.0627  
%DSA-SHA           1.0505    0.0099    1.0523    1.0328    1.0649  
%RSA-SHA1-2        1.0505    0.0095    1.0525    1.033     1.0656  
%RSA-SHA1          1.0506    0.0097    1.0524    1.0324    1.065   
%DSA-SHA1          1.0507    0.0102    1.051     1.0327    1.0654  
%DSA               1.0509    0.0095    1.052     1.0336    1.0644  
%RIPEMD160         1.0741    0.014     1.0746    1.0492    1.0936  
%RSA-RIPEMD160     1.0746    0.0147    1.0761    1.0503    1.0949  
%RSA-SHA           1.0846    0.0161    1.0873    1.0574    1.1073  
%SHA               1.0857    0.0165    1.0863    1.0582    1.1101  
%RSA-SHA224        1.133     0.0246    1.1346    1.0907    1.1695  
%SHA224            1.1331    0.0241    1.1346    1.0911    1.1687  
%SHA256            1.1367    0.0254    1.1369    1.0942    1.1735  
%RSA-SHA256        1.1369    0.0253    1.1369    1.0944    1.174   
%RSA-SHA512        1.2056    0.0383    1.2098    1.1504    1.2561  
%SHA512            1.206     0.0379    1.2113    1.1506    1.2593  
%SHA384            1.2106    0.0397    1.2164    1.1521    1.2651  
%RSA-SHA384        1.2108    0.0396    1.2163    1.1525    1.2635  
%whirlpool         1.327     0.0589    1.3344    1.2401    1.4065

%SELECT cipher, digest, ROUND(duration/(SELECT MIN(duration) FROM cip_dig_tt), 4) FROM cip_dig_tt ORDER BY duration ASC LIMIT 15;
%BF-CBC & MD4 & 1.0
%BF-CBC & RSA-MD4 & 1.0027
%BF-CBC & MD5 & 1.0102
%BF-CBC & RSA-MD5 & 1.0114
%CAMELLIA-128-CBC & MD4 & 1.0126
%CAMELLIA-128-CBC & RSA-MD4 & 1.0133
%CAMELLIA-128-CBC & MD5 & 1.022
%CAMELLIA-128-CBC & RSA-MD5 & 1.0237
%BF-CBC & ecdsa-with-SHA1 & 1.0596
%CAMELLIA-256-CBC & MD4 & 1.0599
%BF-CBC & SHA1 & 1.0612
%BF-CBC & RSA-SHA1 & 1.0613
%CAMELLIA-192-CBC & MD4 & 1.0615
%BF-CBC & RSA-SHA1-2 & 1.0616
%BF-CBC & DSA & 1.0617

\begin{frame}
	\frametitle{Záver}
	\begin{itemize}
		\item kompresia zero vs. random
		\item najrýchlejšia šifra: BF-CBC (AES, 3DES o 30-60\% pomalšie)
		\item najrýchlejší haš: MD4, MD5 (ale bezpečnosť...)
		\item vhodný kompromis: SHA1 (5\% pomalšie ako MD4)
		\item OpenVPN default: BF-CBC, SHA1 :-)
	\end{itemize}

	\begin{columns}
		\tiny
		\column[t]{0.4\textwidth}
			\begin{tabular}{|r|l|l|l|}
				\hline
					\ & šifra & haš & čas \\ \hline
1 & BF-CBC & MD4 & 1.0 \\ \hline
2 & BF-CBC & RSA-MD4 & 1.0027 \\ \hline
3 & BF-CBC & MD5 & 1.0102 \\ \hline
4 & BF-CBC & RSA-MD5 & 1.0114 \\ \hline
\dots & \dots & \dots & \dots \\ \hline
11 & BF-CBC & SHA1 & 1.0612 \\ \hline
			\end{tabular}
		\column[t]{0.5\textwidth}
			\begin{tabular}{|r|l|l|l|}
				\hline
					\ & šifra & haš & čas \\ \hline
\dots & \dots & \dots & \dots \\ \hline
396 & DES-EDE3-CBC & SHA384 & 1.9069 \\ \hline
397 & DES-EDE3-CBC & RSA-SHA384 & 1.9082 \\ \hline
398 & AES-256-CBC & whirlpool & 1.9115 \\ \hline
399 & DES-EDE-CBC & whirlpool & 2.051 \\ \hline
400 & DES-EDE3-CBC & whirlpool & 2.053 \\ \hline
			\end{tabular}
	\end{columns}

	\bigskip
	\small
	Odkazy:
	\footnotesize
	\begin{itemize}
		\item \href{https://github.com/laciKE/openvpn\_test/}{https://github.com/laciKE/openvpn\_test/}
		\item \href{https://openvpn.net/index.php/open-source/documentation/miscellaneous/78-static-key-mini-howto.html}{https://openvpn.net/index.php/open-source/documentation/miscellaneous/78-static-key-mini-howto.html}
	\end{itemize}
\end{frame}


\end{document}
