\documentclass[12pt,a4paper]{article}

\usepackage[slovak]{babel}
\usepackage[utf8]{inputenc}
\usepackage{listings}
\usepackage{graphicx}
\usepackage{tabularx}
\usepackage{amsmath}
\usepackage{amssymb}
\usepackage{hyperref}
\usepackage{multicol}

\usepackage{epstopdf}
\linespread{1.5}

\lstset{
language=python
,breaklines=true
,basicstyle=\ttfamily
, showstringspaces=false}

\textwidth 6.5in
\oddsidemargin 0.0in
\evensidemargin 0.0in

\begin{document}

\thispagestyle{empty}
\begin{center}
    \large{
        \textbf{
            Fakulta matematiky, fyziky a informatiky\\
            Univerzita Komenského, Bratislava
        }
    }
\end{center}

\vspace{2cm}

\begin{figure}[!h]
    \centering
    \includegraphics[width=3.5cm]{komlogo-new.pdf}
\end{figure}

\vspace{1cm}

\begin{center}
    \large{
        \textbf{
            Experimentálne porovnanie výkonu OpenVPN pri~použití rôznych kryptografických algoritmov
        }\\

        2-INF-262	Bezpečnosť IT infraštruktúry \\
        (Seminárna práca)\\
        \url{https://github.com/laciKE/openvpn\_test/}

        \vspace{1.5cm}

        \textbf{
            Ladislav Bačo, Michal Petrucha
        } \\
        Vedúci práce: RNDr. Jaroslav Janáček, PhD.

    }
\end{center}

\vfill

\begin{multicols}{2}
    \begin{flushleft}
        \textbf{\today}
    \end{flushleft}
    \begin{flushright}
        \textbf{Bratislava}
    \end{flushright}
\end{multicols}

\newpage

\section{Úvod}

OpenVPN je populárny multiplatformový nástroj na vytváranie virtuálnych
privátnych sietí.
TODO: omáčka a~formulácia zadania

\section{Spôsob merania}
TODO: lepší názov pre sekciu

\subsection{Nastavenia OpenVPN}

\subsection{Nástroje a~dáta}

\subsection{Použité prostredie}

\subsection{Postretnutú komplikácie}

\section{Výsledky}

\subsection{Vplyv kompresie}

\subsection{Porovnanie šifrovacích algoritmov}

\subsection{Porovnanie hašovacích algoritmov}

\section{Záver}

%\renewcommand{\refname}{Literatúra}
%\phantomsection
%\addcontentsline{toc}{section}{Literatúra}
%\begin{thebibliography}{99}
%  \bibitem{proj} \url{https://github.com/laciKE/openvpn_test} - Repozitár projektu.
%\end{thebibliography}
\end{document}
