\documentclass[12pt,a4paper]{article}

\usepackage[slovak]{babel}
\usepackage[utf8]{inputenc}
\usepackage{listings}
\usepackage{graphicx}
\usepackage{tabularx}
\usepackage{amsmath}
\usepackage{amssymb}
\usepackage{hyperref}
\usepackage{multicol}
\usepackage{fancyvrb}

\usepackage{epstopdf}
%\linespread{1.5}

\lstset{
language=python
,breaklines=true
,basicstyle=\ttfamily
, showstringspaces=false}

\textwidth 6.5in
\oddsidemargin 0.0in
\evensidemargin 0.0in

\begin{document}

\thispagestyle{empty}
\begin{center}
    \large{
        \textbf{
            Fakulta matematiky, fyziky a informatiky\\
            Univerzita Komenského, Bratislava
        }
    }
\end{center}

\vspace{2cm}

\begin{figure}[!h]
    \centering
    \includegraphics[width=3.5cm]{komlogo-new.pdf}
\end{figure}

\vspace{1cm}

\begin{center}
    \large{
        \textbf{
            Experimentálne porovnanie výkonu OpenVPN pri~použití rôznych kryptografických algoritmov
        }\\

        2-INF-262	Bezpečnosť IT infraštruktúry \\
        (Seminárna práca)\\
        \url{https://github.com/laciKE/openvpn\_test/}

        \vspace{1.5cm}

        \textbf{
            Ladislav Bačo, Michal Petrucha
        } \\
        Vedúci práce: RNDr. Jaroslav Janáček, PhD.

    }
\end{center}

\vfill

\begin{multicols}{2}
    \begin{flushleft}
        \textbf{\today}
    \end{flushleft}
    \begin{flushright}
        \textbf{Bratislava}
    \end{flushright}
\end{multicols}

\newpage

\section{Úvod}

OpenVPN \cite{openvpn} je populárny multiplatformový open-source nástroj
na vytváranie virtuálnych privátnych sietí. Na zaručenie dôvernosti
a~integrity dát tečúcich cez takúto sieť sú použité kryptografické
prostriedky, ktoré vyžadujú netriviálne výpočty. V~prípade dôvernosti ide
o~šifrovací algoritmus, na integritu a~autenticitu správ využíva OpenVPN
funkciu HMAC v~kombinácii s~voliteľnou hašovacou funkciou.

Cieľom tejto práce je experimentálne určiť vplyv použitých nastavení
OpenVPN na priepustnosť vytvorenej siete. Konkrétne ide o~použitý
šifrovací algoritmus, použitá hašovacia funkcia a~vplyv použitia kompresie
na prenesené dáta.

\section{Spôsob merania}

V~tejto sekcii popíšeme metódy použité pri meraní a~prostredie, v~ktorom
boli merania vykonané.

\subsection{Nastavenia OpenVPN}

OpenVPN poskytuje viacero spôsobov, ktorými sa vytvára spojenie medzi
koncovými uzlami, medzi ktorými OpenVPN vytvára tunel. Tradičný spôsob
využíva PKI na overenie autenticity účastníkov a~vygenerovanie
symetrických kľúčov pre konkrétnu session.

Druhá možnosť je vopred vygenerovať priamo symetrické kľúče a~cez bezpečný
kanál ich distribuovať medzi účastníkov. Pri tejto metóde odpadá nutnosť
vytvárania certifikátov, ale je obmedzená na práve dvoch účastníkov, na
rozdiel od PKI, kedy môže jeden server obsluhovať koľkokoľvek klientov.

Keďže v~tejto práci využívame iba dvoch účastníkov, pre jednoduchosť sme
zvolili statický zdieľaný kľúč podľa \cite{openvpn-static}.

Merali sme vplyv troch parametrov na priepustnosť: šifrovací algoritmus,
hašovacia funkcia a~použitie kompresie. Pre každú kombináciu týchto
parametrov sme vytvorili dvojicu konfiguračných súborov pre OpenVPN podľa
šablóny \ref{fig:openvpn-config-templates}.

Zoznam podporovaných šifrovacích algoritmov a~hašovacích funkcií je možné
získať prí\-kaz\-mi \verb|openvpn --show-ciphers| a~\verb|openvpn --show-digests|.
Použitá verzia OpenVPN podporuje 58 rôznych šifrovacích algoritmov a~25
hašovacích funkcií, čo vyžadovalo vysokú mieru automatizácie.

\begin{figure}
    \begin{minipage}{.49\linewidth}\centering
        \begin{BVerbatim}
dev tun
remote $SERVER_IP
ifconfig $CLIENT_VPN_IP \
         $SERVER_VPN_IP
secret $STATIC_KEY
auth $DIGEST
cipher $CIPHER
comp-lzo [yes|no]
        \end{BVerbatim}
    \end{minipage}
    \begin{minipage}{.49\linewidth}\centering
        \begin{BVerbatim}
dev tun
ifconfig $SERVER_VPN_IP \
         $CLIENT_VPN_IP
secret $STATIC_KEY
auth $DIGEST
cipher $CIPHER
comp-lzo [yes|no]
        \end{BVerbatim}
    \end{minipage}
    \caption{Šablóny konfiguračných súborov pre OpenVPN klienta (vľavo)
             a~server (vpravo).}
    \label{fig:openvpn-config-templates}
\end{figure}

\subsection{Nástroje a~dáta}

Keďže jedným z~cieľov je zistiť, ako sa prejavuje kompresia, je
očakávateľné, že výsledky sa budú líšiť v~závislosti od toho, či prenášané
dáta sú dobre komprimovateľné, alebo nie. Za týmto účelom sme zvolili dva
súbory: jeden obsahujúci náhodné dáta a~jeden obsahujúci iba nuly.

Pri prvotných testoch sa ukázalo, že systémový pseudonáhodný generátor
\verb|/dev/urandom| na testovacích počítačoch stíhal produkovať výstup
rýchlosťou najviac okolo 5 MB/s, preto bolo potrebné náhodný súbor vopred
vygenerovať. Keďže rýchlosť čítania z~tradičných platňových diskov iba
zriedkavo dosahuje hodnoty blížiace sa ku 100 MB/s, aby sme vylúčili vplyv
rýchlosti čitania z~disku, testovacie súbory boli uložené v~ramdisku.

Veľkosť súborov bola zvolená na 320 MB.

Na prenos súborov po sieti bol použitý nástroj netcat. Každý súbor bol
prenesený raz od klienta na server a~raz zo servera ku klientovi, pričom
príjemca vždy súbor zahodil (\verb|/dev/null|). Keďže je pomerne náročné
spoľahlivo merať čas prenosu na strane príjemcu iba s~použitím
shellscriptu, merali sme iba čas behu \verb|nc| na strane odosielateľa,
pričom na tejto strane \verb|nc| skončilo akonáhle odoslalo posledný byte
vstupu.

Na meranie času sme použili príkaz \verb|date '+%s.%N'| tesne pred a~tesne
po príkaze \verb|nc|, čo je výrazne jednoduchšie, než parsovať výstup
príkazu \verb|time|.

\subsection{Použité prostredie}

Na testovanie bola použitá dvojica identických počítačov s~procesorom
Intel Pentium 4 s~taktovacou frekvenciou 2.80GHz, 1GB RAM a~operačným
systémom Debian GNU/Linux 7.4. Počítače sú spojené 1 GBit sieťou, čo
v~kombinácii so starším procesorom zaručilo, že prenosová rýchlosť bola
pri všetkých testoch obmedzená výpočtovým výkonom a~nie kapacitou linky.

Použitá verzia OpenVPN je 2.2.1 a~keďže OpenVPN využíva implementácie
kryptografických algoritmov z~OpenSSL, použitá verzia tejto knižnice je
1.0.1e-2+deb7u6.

Použité počítače nedisponujú žiadnymi prostriedkami na akceleráciu
kryptografických operácií, či už periférne moduly, alebo rozšírenia
inštrukčnej sady procesora, preto všetky operácie museli prebehnúť
softwarovo.

\subsection{Postretnuté komplikácie}

Počas púšťania testov sme narazili na dve komplikácie hodné zmienky.

Prvou bolo, že napriek tomu, že OpenVPN hlásilo 58 rôznych podporovaných
šifrovacích algoritmov, v~skutočnosti bolo možné použiť iba 16 z~nich.
Súčasťou špecifikácie šifrovacieho algoritmu totiž je aj použitý blokový
mód a~v~našej konfigurácii OpenVPN umožňovalo použiť iba mód CBC.

Druhou komplikáciou bol zvláštne sa prejavujúci race condition. Po
dokončení všetkých prenosov sa totiž reštartovala služba OpenVPN s~novými
nastaveniami. Na strane odosielateľa sa to ale stávalo hneď po zapísaní
posledného bytu vstupu do socketu, čo bolo občas skôr, než stihol príjemca
spracovať celý súbor. Výsledkom bolo, že na strane príjemcu ešte \verb|nc|
čítalo posledné byty prenášaného súboru, ale na strane odosielateľa bol
zatiaľ VPN tunel ukončený. V~dôsledku tohto ostával proces \verb|nc| na
strane príjemcu visieť v~nedefinovanom stave, snažiac sa čítať vstup zo
socketu, ktorému ale medzičasom prestal existovať sieťový interface, teda
tento proces ostával visieť nadobro.

Tento race condition sme eliminovali uspaním odosielateľa na niekoľko
sekúnd po poslednom prenose.

\section{Výsledky}

Výsledky sme spracúvali s~pomocou databázy SQLite \cite{sqlite}. Namerané
dáta spolu s~SQL dotazmi použitými na vypočítanie štatistík popísaných
v~tejto časti sú dostupné v~projektovom repozitári \cite{proj}.

\subsection{Vplyv kompresie}

Pre určenie vplyvu kompresie na prenosovú rýchlosť sme pre každú trojicu
šifry, hašovacej funkcie a~súboru spočítali pomer medzi časom bez
kompresie a~časom s~kompresiou. Pre náhodný aj nulový súbor sme následne
spravili štatistiku týchto pomerov cez všetky dvojice šifry a~hašovacej
funkcie. Výsledok je možné vidieť v~tabuľke \ref{tbl:compression}.

\begin{table}\centering
    \caption{Vplyv kompresie na čas prenosu; čas prenosu bez kompresie: 1}
    \label{tbl:compression}

    \begin{tabular}{|l||l|l|l|l|l|}
        \hline
        \bf súbor & \bf  avg & \bf stdev & \bf median & \bf min & \bf max \\ \hline 
        random & 1.0027 & 0.002 & 1.0027 & 0.9796 & 1.0075 \\ \hline
        zero & 0.6614 & 0.0943 & 0.6627 & 0.5164 & 0.84\\ \hline
    \end{tabular}
\end{table}

Ako tabuľka naznačuje, pre náhodný súbor, ktorý je veľmi ťažko
komprimovateľný, použitie kompresie prinieslo iba zanedbateľné zrýchlenie
vo veľmi obmedzenom počte prípadov; skôr naopak, väčšinou spôsobilo veľmi
mierne spomalenie.

V~prípade nulového súboru, ktorý je veľmi dobre komprimovateľný, sme
očakávali značné zrýchlenie. Z~tabuľky vyplýva, že zrýchlenie bolo celkom
konzistentne o~približne jednu tretinu, s~relatívne malou disperziou.

\subsection{Porovnanie šifrovacích algoritmov}
\label{sec:cipher-comparison}

Štatistiky pre toto porovnanie sme vytvárali iba na základe prenosu
náhodného súboru bez kompresie.

Najprv sme identifikovali algoritmus Blowfish ako konzistentne
najrýchlejší. Následne sme pre každý hašovací algoritmus vydelili čas
prenosu časom pre šifru Blowfish, čím sme vyjadrili relatívne spomalenie
každého šifrovacieho algoritmu oproti Blowfish pri danej hašovacej funkcii.

Nakoniec sme pre každý šifrovací algoritmus spočítali štatistiku
relatívnych spomalení, ktorú možno vidieť v~tabuľke
\ref{tbl:cipher-comparison}.

\begin{table}\centering
    \caption{Vplyv šifrovacieho algoritmu na čas prenosu; čas prenosu pri použití BF-CBC: 1}
    \label{tbl:cipher-comparison}
    \begin{tabular}{|l||l|l|l|l|l|}
        \hline
        \bf šifra & \bf  avg & \bf stdev & \bf median & \bf min & \bf max \\ \hline 
        BF-CBC & 1.0 & 0.0 & 1.0 & 1.0 & 1.0\\ \hline
        CAMELLIA-128-CBC & 1.0148 & 0.0036 & 1.014 & 1.0096 & 1.0236 \\ \hline
        CAMELLIA-256-CBC & 1.0573 & 0.0044 & 1.0583 & 1.0456 & 1.0655 \\ \hline
        CAMELLIA-192-CBC & 1.058 & 0.0046 & 1.0599 & 1.0467 & 1.0663 \\ \hline
        CAST5-CBC & 1.0781 & 0.0078 & 1.0799 & 1.0603 & 1.0894 \\ \hline
        DES-CBC & 1.1026 & 0.0119 & 1.1034 & 1.0776 & 1.1402 \\ \hline
        SEED-CBC & 1.1045 & 0.0068 & 1.1058 & 1.0852 & 1.1147 \\ \hline
        DESX-CBC & 1.1134 & 0.0109 & 1.1159 & 1.0887 & 1.1321 \\ \hline
        AES-128-CBC & 1.2823 & 0.0221 & 1.2867 & 1.2235 & 1.3078 \\ \hline
        AES-192-CBC & 1.3687 & 0.0293 & 1.376 & 1.2925 & 1.4016 \\ \hline
        RC2-64-CBC & 1.4211 & 0.0364 & 1.4294 & 1.3309 & 1.4679 \\ \hline
        RC2-40-CBC & 1.4213 & 0.0368 & 1.4306 & 1.3314 & 1.4694 \\ \hline
        RC2-CBC & 1.4213 & 0.0365 & 1.4315 & 1.3318 & 1.4687 \\ \hline
        AES-256-CBC & 1.4553 & 0.0376 & 1.4671 & 1.3593 & 1.498 \\ \hline
        DES-EDE-CBC & 1.5821 & 0.0508 & 1.5934 & 1.4585 & 1.6539 \\ \hline
        DES-EDE3-CBC & 1.5824 & 0.0504 & 1.592 & 1.4599 & 1.6551 \\ \hline
    \end{tabular}
\end{table}

Ako najpomalší algoritmus sa ukázal trojitý DES vo variantoch s~jedným
a~tromi kľúčmi. AES a~RC2 boli tiež výrazne pomalšie, než Blowfish, zatiaľ
čo Camellia je iba nepatrne pomalšia oproti Blowfish.

\subsection{Porovnanie hašovacích funkcií}

Pri tomto porovnaní sme opäť vychádzali iba z~časov prenosu náhodného
súboru bez kompresie.

Postup pri vytváraní štatistík bol podobný ako v~časti
\ref{sec:cipher-comparison}, s~tým rozdielom, že tentokrát sme
identifikovali ako najrýchlejšiu funkciu MD4 a~časy prenosu sme
normalizovali na relatívne spomalenie oproti tejto funkcii.
Výsledok je v~tabuľke \ref{tbl:digest-comparison}.

\begin{table}\centering
    \caption{Vplyv hašovacej funkcie na čas prenosu; čas prenosu pri použití MD4: 1}
    \label{tbl:digest-comparison}
    \begin{tabular}{|l||l|l|l|l|l|}
        \hline
        \bf haš & \bf  avg & \bf stdev & \bf median & \bf min & \bf max \\ \hline 
        MD4 & 1.0 & 0.0 & 1.0 & 1.0 & 1.0 \\ \hline
        RSA-MD4 & 1.0027 & 0.0066 & 1.0013 & 0.9985 & 1.0272 \\ \hline
        MD5 & 1.0082 & 0.0019 & 1.0084 & 1.0039 & 1.0111 \\ \hline
        RSA-MD5 & 1.0092 & 0.002 & 1.0092 & 1.005 & 1.0114 \\ \hline
        ecdsa-with-SHA1 & 1.0494 & 0.0098 & 1.0509 & 1.0312 & 1.0643 \\ \hline
        DSA-SHA1-old & 1.0498 & 0.0094 & 1.0512 & 1.0328 & 1.0631 \\ \hline
        SHA1 & 1.0502 & 0.0094 & 1.0507 & 1.032 & 1.0627 \\ \hline
        DSA-SHA & 1.0505 & 0.0099 & 1.0523 & 1.0328 & 1.0649 \\ \hline
        RSA-SHA1-2 & 1.0505 & 0.0095 & 1.0525 & 1.033 & 1.0656 \\ \hline
        RSA-SHA1 & 1.0506 & 0.0097 & 1.0524 & 1.0324 & 1.065 \\ \hline
        DSA-SHA1 & 1.0507 & 0.0102 & 1.051 & 1.0327 & 1.0654 \\ \hline
        DSA & 1.0509 & 0.0095 & 1.052 & 1.0336 & 1.0644 \\ \hline
        RIPEMD160 & 1.0741 & 0.014 & 1.0746 & 1.0492 & 1.0936 \\ \hline
        RSA-RIPEMD160 & 1.0746 & 0.0147 & 1.0761 & 1.0503 & 1.0949 \\ \hline
        RSA-SHA & 1.0846 & 0.0161 & 1.0873 & 1.0574 & 1.1073 \\ \hline
        SHA & 1.0857 & 0.0165 & 1.0863 & 1.0582 & 1.1101 \\ \hline
        RSA-SHA224 & 1.133 & 0.0246 & 1.1346 & 1.0907 & 1.1695 \\ \hline
        SHA224 & 1.1331 & 0.0241 & 1.1346 & 1.0911 & 1.1687 \\ \hline
        SHA256 & 1.1367 & 0.0254 & 1.1369 & 1.0942 & 1.1735 \\ \hline
        RSA-SHA256 & 1.1369 & 0.0253 & 1.1369 & 1.0944 & 1.174 \\ \hline
        RSA-SHA512 & 1.2056 & 0.0383 & 1.2098 & 1.1504 & 1.2561 \\ \hline
        SHA512 & 1.206 & 0.0379 & 1.2113 & 1.1506 & 1.2593 \\ \hline
        SHA384 & 1.2106 & 0.0397 & 1.2164 & 1.1521 & 1.2651 \\ \hline
        RSA-SHA384 & 1.2108 & 0.0396 & 1.2163 & 1.1525 & 1.2635 \\ \hline
        whirlpool & 1.327 & 0.0589 & 1.3344 & 1.2401 & 1.4065 \\ \hline
    \end{tabular}
\end{table}

V~jednom prípade prebehol prenos najrýchlejšie pri použití funkcie MD4
v~kombinácii s~podpisom pomocou RSA, čo je dôvod pre minimum menšie ako 1
v~druhom riadku tabuľky. Vo všetkých ostatných prípadoch samotná funkcia
MD4 prebehla najrýchlejšie.

Z~nám neznámych dôvodov knižnica OpenSSL poskytuje aplikáciám hašovacie
funkcie v~kombinácii s~podpisovými schémami RSA a~DSA, pričom z~hľadiska
aplikačného rozhrania ich neodlišuje od klasických hašovacích funkcií.
Nepodarilo sa nám zistiť, aký kľúč sa pri takýchto funkciách používa, ani
akú výhodu majú oproti samotným hašovacím funkciám (keďže sú používané
čisto ako hašovacie funkcie, nie na podpis).

Každopádne, podpísanie výsledku sa vo všetkých prípadoch prejavilo iba
minimálne, niekedy dokonca prebehlo rýchlejšie, než bez podpisu.

Funkcie MD4 a~MD5 boli suverénne najrýchlejšie, ibaže v~dnešnej dobe sa
už neodporúča používať ktorúkoľvek funkciu v~nových protokoloch
a~aplikáciách \cite{rfc-md4-obsolete, rfc-hmac-md5}.

Funkcia SHA1 sa ukázala byť iba mierne pomalšia. Keďže podľa našich
informácií pre túto funkciu doteraz nevyšlo odporúčanie nepoužívať ju ako
hašovaciu funkciu pre HMAC, javí sa táto funkcia ako dobrý kandidát
poskytujúci dostatočnú bezpečnosť bez výrazného negatívneho vplyvu na
výkon. RIPEMD160 je tiež vhodný kandidát, ktorý je iba nepatrne pomalší
oproti SHA1, pre ktorý nie je zatiaľ známy žiaden útok.

Varianty SHA s~väčšou dĺžkou bloku sa ukázali ako nezanedbateľne pomalšie
a~Whirlpool sa umiestnil na poslednom mieste.

\section{Záver}

\renewcommand{\refname}{Literatúra}
\phantomsection
\addcontentsline{toc}{section}{Literatúra}
\begin{thebibliography}{99}
  \bibitem{proj} \url{https://github.com/laciKE/openvpn_test} -- Repozitár projektu.
  \bibitem{openvpn} \url{http://openvpn.net/index.php/open-source.html} -- Domovská stránka OpenVPN.
  \bibitem{openvpn-static} \url{http://openvpn.net/index.php/open-source/documentation/miscellaneous/78-static-key-mini-howto.html} -- Návod na jednoduché nastavenie statického zdieľaného kľúča pre OpenVPN.
  \bibitem{sqlite} \url{http://sqlite.org/} -- Domovská stránka SQLite.
  \bibitem{rfc-md4-obsolete} \url{http://tools.ietf.org/html/rfc6150} -- RFC 6150: MD4 to Historic Status.
  \bibitem{rfc-hmac-md5} \url{http://tools.ietf.org/html/rfc6151} -- RFC 6151: Updated Security Considerations for the MD5 Message-Digest and the HMAC-MD5 Algorithms.
\end{thebibliography}
\end{document}
